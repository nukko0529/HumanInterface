\documentclass[dvipdfmx]{ujarticle}

%パッケージ
\usepackage{amssymb,amsmath}
\usepackage[dvipdfmx]{graphicx}

%レイアウト
\setlength{\oddsidemargin}{0mm}
\setlength{\textwidth}{160mm}
\setlength{\topmargin}{-9mm}
\setlength{\headheight}{0mm}
\setlength{\textheight}{241mm}

%定理環境
\usepackage{theorem}\theorembodyfont{\normalfont}
\newtheorem{definition}{定義}[section]
\newtheorem{lemma}[definition]{補題}
\newtheorem{theorem}[definition]{定理}
\newtheorem{corollary}[definition]{系}
\newtheorem{proposition}[definition]{命題}
\newtheorem{example}[definition]{例}
\newtheorem{proof}{証明:\hspace*{-3pt}}
\renewcommand{\theproof}{}
\def\QED{\hspace*{\fill}$\square$}

\begin{document}

%%% Header

\title{B3ヒューマンインターフェース期末レポート}
\author{1223033129 温水心琴}
\date{\today}

\maketitle

%%% Body

\section{導入}
\subsection{設問}
与えられた4つの設問のうち,一つ目にあたる特異な視覚ディスプレイについて述べる.
このレポートでは,特異な視覚ディスプレイとして網膜投影ディスプレイを例に挙げる.

\subsection{網膜投影型ディスプレイとは}
網膜投影型ディスプレイは,マックスウェル視を利用して観察者の眼球内に光を入れることで映像を呈示することができる.
現実世界に仮想映像を表示するARにおいて,シースルー型HMDが多用されているが,
シースルー型HMDで現実世界と仮想映像を同時に観察しようとすると,現実世界を観察するときの
焦点調節はダイナミックに変化するにも関わらず仮想世界を見るときの焦点調節は固定されているため,
焦点調節と輻輳が一致しないという問題が生じる.
一方で,網膜投影型ディスプレイでは,この問題を克服できると期待される.
また,ARにおいて近点から無限遠まで外界を観察する場合でも,仮想映像が常に鮮明に見えるという利点もあると考えられる.

\cite{3DIC2000}

\section{網膜投影型ディスプレイの原理と技術}
\subsection{マクスウェル視}
通常,物体を観察するときは対象物体からくる書く参考を水晶体で収束して網膜上で結像する.
一方,マクスウェル視は,物体から出た光を一度レンズで集光させ,その周口店に瞳孔中心をおいて網膜上に投影して像を観察する.
瞳孔の中心を通る光は,水晶体の厚みの変化に関係なく網膜の広い範囲に均一に達する.
そのため,マクスウェル視で物体を見た場合,水晶体の調節機能を使用せず,映像を直接網膜に投影するため,
現実のどの位置に焦点を合わせても仮想映像を鮮明に観察することができる.
また,観察者は焦点深度の深い映像を観察できる.

\cite{FIT2002}
\cite{RetinalPSD}
\cite{ITE2014}

\subsection{HOE}
マックスウェル視を利用して映像を呈示するに当たって,
平行光線束を収束させるレンズと外界光に仮想映像を表示させるコンバイナが必要である.
コンバイナとして多用されるハーフミラーでは,外界と電子映像のどちらも明るく観察できない,
という問題がある.
その問題を解決するのがHOEという,ホログラム乾板に2つのレーザー光を照射して
生成される漢商高を記録した回折格子である.
多機能であり,レンズ機能,干渉フィルターなどの役割を果たす.
HOEは,外界の全波長域の光を透過させると同時に特定波長の光を解説指せる
波長選択性を持つため,外界光と仮想映像を同時に明るく観察できる.
また,特定方向から入射した光のみを回折する角度選択性を持つため,
左右の瞳孔にそれぞれマクスウェル視として光を呈示することができる.

\cite{3DIC2000}
\cite{RetinalPSD}
\cite{ITE2014}

\subsection{網膜投影型ディスプレイの基本原理}
網膜投影型ディスプレイは,画像信号に応じて強度変調された光束を直接網膜上でラスタ走査して画像を形成する.
光源部で赤,青,緑のレーザー光を強度変調し,3色分の光線を光軸をそろえて一本の光束として出力する.
光走査部は光源部からの光線を2次元的に偏向走査する.
偏向走査された光は伝播にしたがってひろがり,再度収束して観察者の瞳孔に入射する.
その点像入射角を光束に変化させることで画像が形成される.
また,この点像の位置変化と,光束の色や明るさの変調を同期させることで
網膜上の任意の位置に任意の色と明るさの点像を形成することができる.

\cite{ITE2011}

\section{網膜投影型ディスプレイの利点と課題}
\subsection{利点}
一般的なシースルー型HMDの構成では,ハーフミラーなどを使用することで
ディスプレイの表示画像と観察者の周囲の風景を重ね合わせる.
そのため,仮想映像の表示位置が固定されてしまい,仮想映像と現実映像の両方に
同時に目の焦点を合わせることができないという問題が生じる.

一方で,網膜投影型ディスプレイはマクスウェル視の原理を用いて瞳孔を通して映像を直接網膜に投影する.
そのため,目の焦点位置に関わらず,視野の一部に重なって映像が表示され,仮想画像が常に鮮明に見える.
また,深い焦点深度を実現できるため,見る対象に応じて
頻繁に焦点を調節する必要がなくなり,眼精疲労が生じにくくなる.

\cite{FIT2002}
\cite{ITE2011}
\cite{MedApp}
\cite{ITE2016}

日常において知覚する立体感は,焦点調節と輻輳が一致している状態であるが,
立体像表示方式の主流となっている両眼視差方式では,それらが一致しないため,疲労感を伴うなどの問題点がある.
また,両眼視差方式の立体ディスプレイでは,輻輳を変化させる範囲が狭く,そのため
奥行方向のダイナミックレンジが狭い立体表現になっていたと考えられる.

網膜投影型ディスプレイは,水晶体の焦点調節機能の影響を受けずに映像を知覚できるため,
輻輳を変化させる範囲が広くなる.

\cite{RetinalPSD}

さらに,ハーフミラーが用いられるシースルー型HMDの欠点として,外界光の透過と仮想映像の反射が
トレードオフの関係にあるため,外界と仮想映像のどちらも明るく観察することができない,というものが挙げられる.

波長選択性を持つHOEを利用することで,網膜投影型ディスプレイでは外界光と仮想映像を
同時に明るく観察することができる.

\cite{3DIC2000}

また,画面の小型・軽量化と高精細化は互いにトレードオフの関係にあり,一般に表示を
公正化するためには画面を大きくする必要がある.
網膜投影型は比較的小型に実装しやすく,さらに光の走査を低電力で行う事ができれば
この問題の解決にも繋がると考えられる.

\subsection{課題}
マクスウェル視の原理を応用して網膜に直接光を当てているため,光の収束点が観察者の瞳孔から
ずれてしまうと映像が観察できなくなる.
マクスウェル視は元より,原理上瞳孔の中心と集光点の位置合わせが難しいこともあり,
視域が狭い,という問題がある.

光線の収束点の位置を瞳孔上ではなく眼球内部に設定する,収束点の位置を可変にする,
などの対策を取ることでこの課題の解決が見込まれる.
収束点の位置を可変にできると,眼球の移動によって瞳孔の位置が変化した場合でも,
収束点と瞳孔の位置が重なった状態を保つことが可能になる.
これによって,ディスプレイの視域がひろがり,観察者が常に網膜投影方式の画像を見ることができる.

\cite{FIT2002}
\cite{ITE2011}
\cite{ITE2016}

\section{網膜投影型ディスプレイの応用例}
\subsection{視力補助のための利用}
従来の視力補助器具は,各人の近視,遠視の度合を補正しながら文字の拡大などを行うことはできなかった.

しかし,網膜投影型ディスプレイは,水晶体の調節機能を使用せずに映像を知覚することができるため,
視力に関係なく網膜上により明瞭な映像が表示される.

さらなる応用方法として,病気で障害された部分を使わず,残った部分に像を写すことで,
失われた視覚機能を補うことができる可能性がある.
例えば,成人病としての緑内障,加齢性の黄斑変性症,網膜色素変性症などに対して,
残された網膜に映像を投影することが可能になる.

眼球内部の後半部はすべて網膜であり受光特性を持つが,人間はその内で優れた感度と分解能を持つ
黄斑のみを使用している.
そのため,一般にこの部分が傷められた場合視力に問題が生じるが,残っている網膜部に
映像を投射することで映像を知覚できる.

手持ち型,メガネ型など様々な形で視力補助のための網膜投影ディスプレイが開発されてきている.

\cite{ITE2011}
\cite{MedApp}

ソニーが発売した網膜投影カメラキット『DSC-HX99 RNV kit』は,デジタルスチルカメラ サイバーショット®『DSC-HX99』と,
株式会社QDレーザのレーザ網膜投影技術を応用したビューファインダー『RETISSA NEOVIEWER (レティッサ ネオビューワ)』を
組み合わせた商品である.
この商品は,目のピンと調節能力の影響を受けにくいレーザー網膜投影方式を利用することで,
従来のカメラでは撮影したい映像が見づらかった人が,カメラが
とらえる景色を網膜に投影し,写真や動画の撮影を行うことができる.

\cite{sony}

\subsection{3次元表示}
超多眼の状態を再現することで,網膜投影式であっても3次元映像を表示することができる.
超多眼状態は,眼球瞳孔内に2本以上の光線,視差画像が入射する状態で,立体視の要因の
1つである水晶体の調節を誘導する.
マクスウェル視の原理を応用し,複数の視差画像を瞳孔上の異なる位置に収束させることで
超多眼の条件を満たす.
各画素を網膜上に鮮明に投影しながらも,網膜上の重畳度合でピントが合う部分と
ぼける部分を再現することで,3次元画像のように知覚する.

\cite{ITE2014}
\cite{ITE2016}

%%% Reference
\begin{thebibliography}{99}
    \bibitem{3DIC2000} 安東孝久,濱岸五郎,坂東進,志水英二.2眼立体視型網膜投影ディスプレイ.
        3D Image Conference.2000,4-6,p103-106
    \bibitem{FIT2002} 江泉清隆,髙橋秀也,志水英二.網膜投影型表示システムの一方式.
        FIT(情報科学技術フォーラム).2002,K-34,p435-436
    \bibitem{RetinalPSD} 安東孝久.網膜投影型立体ディスプレイ.
        光学.2002,31巻,3号,p157-159
    \bibitem{ITE2014} 高塚康公,吉本佳世,高橋秀也.超多眼方式を用いた網膜投影型ヘッドマウント3次元ディスプレイ.
        映像情報メディア学会年次大会.2014,23-9
    \bibitem{ITE2011} 志水英二.網膜走査・投影方式ディスプレイ.
        映像情報メディア学会誌.2011,65巻,6号,p758-763
    \bibitem{MedApp} 菅原充.網膜投影型レーザアイウェア技術:医療福祉応用からスマートグラスまで.
        第80回応用物理学会秋季学術講演会 講演予稿集.2019
    \bibitem{ITE2016} 髙橋秀也.網膜投影型ヘッドマウント3次元ディスプレイ.
        ITE Technical Report.2016,40巻,20号,p15-19
    \bibitem{IEEJ2009} 山田恵三,栗山敏秀.非対称シリコン・マイクロミラーを用いためがね型網膜ディスプレイ.
        電学論 E.2009,129巻,2号,p36-40
    \bibitem{sony} ソニー株式会社."ロービジョン者の創作意欲に寄り添う網膜投影カメラを発売".2023/02/21
        https://www.sony.co.jp/corporate/information/news/202302/23-007/
\end{thebibliography}

%ex
%abc abc.defを参照する \cite{hogehoge}.

\end{document}